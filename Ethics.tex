% Please do not change the document class
\documentclass{scrartcl}

% Please do not change these packages
\usepackage[hidelinks]{hyperref}
\usepackage[none]{hyphenat}
\usepackage{setspace}
\usepackage{graphicx}
\usepackage{subcaption}
\doublespace

% You may add additional packages here
\usepackage{amsmath}

% Please include a clear, concise, and descriptive title
\title{COMP230 - Should loot boxes in video games legally be considered gambling?}

% Please put your student number in the author field
\author{1507729}

\begin{document}

\maketitle

\section{An introduction to gambling}
The gambling industry is huge and according to a survey carried out by the UK Gambling Commission, it made £13.8bn between Oct 2015 and Sep 2016 within the Great Britain alone. \cite{GamCom2017Stats} The definition of Gambling this paper will be using is defined by the Oxford dictionary as follows: "Take risky action in the hope of a desired result" \cite{Oxford2017Gamble}. 

It's quite well known that Casinos will use various tricks to make as much money as possible. \cite{Casino2014design, Libby2014Tricks} For the purpose of this paper the one's worth making note of are, maze layouts with the casino as the area you always have to pass through and using their own currency like chips to make customers forget the value of the money they are spending.

\section{Video Games and Loot Drops}
Video games have had random item drops for a long time as shown in video games like Diablo \cite{Blizzard1996Diablo}, Castlevania: Symphony of the Night \cite{Konami1997Castlevania}, Final Fantasy II \cite{Square1988Final}and more recent games like Middle Earth: Shadow of War \cite{Monolith2017Middle} and Assassin's Creed Origins. \cite{Ubisoft2017Assassin} The use of a loot drop system in games isn't uncommon and can be used to keep players playing if they want to get the best items and gear. This is very common practice for many MMORPG games, and the Entertainment Software Rating Board (ESRB) don't consider loot drops to be gambling.

\section{Video Games and Loot Boxes}

With the introduction of loot boxes which can be purchased using an in-game premium currency only available through additional micro transactions. There were calls to ask for the ESRB to update their ratings to cover games that feature loot boxes the ESRB responded later responded saying. “While there’s an element of chance in these mechanics, the player is always guaranteed to receive in-game content (even if the player unfortunately receives something they don’t want). We think of it as a similar principle to collectible card games: Sometimes you’ll open a pack and get a brand new holographic card you’ve had your eye on for a while. But other times you’ll end up with a pack of cards you already have.” \cite{Jason2017ESRB} This raises the question however of when does something change from being an element of chance and being fun to becoming gambling. It's an important distinction that with the example of collectable card games, that these cards have a real world value and rare card could be sold for a profit. Going by their definition of it's not gambling because you get something even if it's of a lower value, then a Casino could say they don't practice gambling by offering a minimum return for every on every bet. Josh Bycer a writer for Gamasutra \cite{Josh2017Loot} suggests that the distinction between something being gambling or not is if it's a reward based as a direct result of user skill or input rather than through pure chance, meaning if a player can affect the loot they get with skill it's not gambling.

\section{The UK Government on Video Game Gambling}
Although the ESRB stated that loot boxes aren't gambling people within the UK started a petition to the UK parliament to ask for the government to pass laws to help regulate loot boxes within the video games industry primarily focusing on games available to children. The initial response on this was to refer to the Gambling Commission who stated their position \cite{GamCom2017position}, who had a discussion on the topic back in 2016 \cite{GamCom2016discussion} Which does admit that there is a "lack of contemporary and directly applicable case law in some of these areas." 

They later go one to state that "In our view, the ability to convert in game items into cash, or to trade them (for other items of value), means they attain a real world value and become articles of money or money’s worth. Where facilities for gambling are offered using such items, a licence is required in exactly the same manner as would be expected in circumstances where somebody uses or receives casino chips as a method of payment for gambling, which can later be exchanged for cash." Meaning that the Gambling Commission does view any game that has their in-game items converted into real world value, something that requires a licence the same way a casino does this however is only for the platform that supplies the conversion rather than the games publishers/producers themselves. Arguably the most notorious examples would be the many different sites for trading Counter-Strike: Global Offensive \cite{valve2012CS} skins. .Though the Gambling Commission can only inform the courts on their findings and views the need for precedent to be set in this area is quite clear and would lay to rest any further debate on the matter.

\section{Considerations for loot boxes as gambling}
If a game features loot boxes, that can be purchased through premium currency it's an decision that was actively made. That means that a company is aware that certain items are more valuable in-game, and that users will want these items. By limiting the chance of getting these items it is gambling, and it's companies should be aware of the risks that loot boxes as a form of gambling could have on their player base. \cite{griffiths2000risk, johansson2009risk, chambers2003developmental} If companies are going to include loot boxes or similar systems for micro transactions they should be aware of the impact that it can have on their player base, especially those younger players who could potentially suffer the most from these practices.

\section{Conclusion}
To conclude, video games are an incredibly popular form of entertainment but also a tool for education. If companies are using loot boxes they could be turning their players into gambling addicts by rewarding those who indulge in their micro transaction systems and punishing players who do not take part, this is not just incredibly predatory for players who are susceptible to these tactics but also can lead to the loss of players due to the penalty for not investing additional funds into their games.

\bibliographystyle{ieeetran}
\bibliography{references}

\end{document}
