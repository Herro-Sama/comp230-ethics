% Please do not change the document class
\documentclass{scrartcl}

% Please do not change these packages
\usepackage[hidelinks]{hyperref}
\usepackage[none]{hyphenat}
\usepackage{setspace}
\doublespace

% You may add additional packages here
\usepackage{amsmath}

% Please include a clear, concise, and descriptive title
\title{COMP230 - Ethics and Professionalism - Does the Video Games Industry encourage Gambling in young audiences}

% Please put your student number in the author field
\author{1507729}

\begin{document}

\maketitle

\section{Proposal}
The games industry is filled with small DLC and microtransactions, these are designed to enhance the gameplay experience of the player by adding or unlocking content. In recent years however microtransactions have started to adapt a loot box system which allows players to pay a small amount to unlock boxes with various game items or cosmetics. A study done by NPD shows that the largest number of purchases are for Weapons/tools for use in-game. 

Many of these microtransactions however are being tailored to maximise the amount that people will spend during their time playing the game. They are using many of the same tactics used within the gambling industry to capitalise on people who have addictive tendencies. With many of the games who use this model designing their game around creating a need for these types of transaction, with games featuring loot crates to give you a chance to get the in-game item you actually want, some of these games having age ratings well below that of the legal age for gambling.

The question is does this count as gambling, because the Oxford Dictionary defines the verb Gambling as the following "Take risky action in the hope of a desired result." Which would include loot boxes, meaning that they should count as gambling.


\bibliographystyle{ieeetran}
\bibliography{references}

\end{document}
